\documentclass[a4paper,11pt,twocolumn]{article}

\usepackage[czech]{babel}
\usepackage[IL2]{fontenc}
\usepackage[utf8]{inputenc}
\usepackage{times}
\usepackage{mathtools}
\usepackage{amsfonts}
\usepackage{amsthm}

\usepackage[left=15mm,text={18cm, 25cm},top=25mm]{geometry}

\setcounter{page}{0}

\theoremstyle{definition}
\newtheorem{definition}{Definice}[section]
\newtheorem{veta}{Věta}

\newtheorem{theorem}[definition]{Algoritmus}

\begin{document}

\thispagestyle{empty}

\onecolumn
\begin{center}
    \Huge
    \textsc{Fakulta informačních technologií \\
            Vysoké učení technické v Brně}\\
    \vspace{\stretch{0.382}}
        \huge Typografie a publikování - 2. projekt\\
        Sazba dokumentů a matematických výrazů\\
    \vspace{\stretch{0.618}}
\end{center}
{\LARGE 2017 \hfill
Róbert Ďurovič}

\newpage

\twocolumn

\section*{Úvod}

V~této úloze si vyzkoušíme sazbu titulní strany, matematických vzorců, prostředí a~dalších textových struktur obvyklých pro technicky zaměřené texty, například rovnice (\ref{eq:reference}) nebo definice \ref{def1} na straně \pageref{def1}.

Na titulní straně je využito sázení nadpisu podle optického středu s~využitím zlatého řezu. Tento postup byl probírán na přednášce.

\section{Matematický text}

Nejprve se podíváme na sázení matematických symbolů a~výrazů v~plynulém textu. Pro množinu $V$ označuje $\mbox{card}(V)$ kardinalitu $V$.
Pro množinu $V$ reprezentuje $V^*$ volný monoid generovaný množinou \textit{V} s~operací konkatenace. Prvek identity ve volném monoidu $V^*$ značíme symbolem $\varepsilon$. Nechť $V^+ = V^* - \{\varepsilon\}$. Algebraicky je tedy $V^+$ volná pologrupa generovaná množinou $V$ s~operací konkatenace. Konečnou neprázdnou množinu $V$ nazvěme $abeceda$. Pro $\omega \in V^*$ označuje $|\omega|$ délku řetězce $\omega$. Pro $W \subseteq V$ označuje $\mbox{occur}(\omega, W)$ počet výskytů symbolů z~$W$ v~řetězci $\omega$ a~$\mbox{sym}(\omega, i)$ určuje $i$-tý symbol řetězce $\omega$; například $\mbox{sym}(abcd, 3) = c$.

Nyní zkusíme sazbu definic a~vět s~využitím balíku {\ttfamily amsthm}.

\begin{definition} \label{def1} 
\emph{Bezkontextová gramatika} je čtveřice $G = (V, T, P, S)$, kde $V$ je totální abeceda, $T \subseteq V$ je abeceda terminálů, $S \in (V\,-\, T)$ je startující symbol a~$P$ je konečná množina \emph{pravidel}
tvaru $q\!\!: A \to \alpha$, kde $A \in (V-T)$, $\alpha \in V^*$ a~$\,q\,$ je návěští tohoto pravidla. Nechť $N = V\,-\,T$ značí abecedu neterminálů.
Pokud $q\!\!: A \to \alpha \in P, \gamma, \delta \in V^*$, $G$ provádí derivační krok z~$\gamma A \delta$ do $\gamma \alpha \delta$ podle pravidla $q\!\!: A \to \alpha$, symbolicky píšeme 
$\gamma A \delta \Rightarrow \gamma \alpha \delta \quad [q: A \to \alpha]$ nebo zjednodušeně $\gamma A \delta \Rightarrow \gamma \alpha \delta$. Standardním způsobem definujeme $\Rightarrow ^m$, kde $m \geq 0$. Dále definujeme tranzitivní uzávěr $\Rightarrow^+$ a~tranzitivně-reflexivní uzávěr $\Rightarrow^*$.
\end{definition}
Algoritmus můžeme uvádět podobně jako definice textově, nebo využít pseudokódu vysázeného ve~vhodném prostředí (například {\ttfamily algorithm2e}).

\begin{theorem} \label{algo} 
\textit{Algoritmus pro ověření bezkontextovosti gramatiky. Mějme gramatiku $G = (N, T, P, S)$.
    \begin{enumerate}
        \item \label{krok} Pro každé pravidlo  proveď test, zda $p \in P$ na levé straně obsahuje právě jeden symbol z~$N$.
        \item Pokud všechna pravidla splňují podmínku z~kroku \ref{krok}, tak je gramatika $G$ bezkontextová.
    \end{enumerate}}
\end{theorem}

\begin{definition} \label{def2}
    \emph{Jazyk} definovaný gramatikou $G$ definujeme jako $L(G) = \{\omega \in T^*|S \Rightarrow^* \omega\}$ .
\end{definition}

\subsection{Podsekce obsahující větu}

\begin{definition} \label{def3} 
Nechť $L$ je libovolný jazyk. $L$ je \emph{bezkontextový jazyk}, když a~jen když $L = L(G)$, kde $G$ je libovolná bezkontextová gramatika.
\end{definition}

\begin{definition} \label{def4}
Množinu $\mathcal{L}_{CF} = \{L|L$ je bezkontextový jazyk\} nazýváme \emph{třídou bezkontextových jazyků}.
\end{definition}

\begin{veta} \label{veta}
Nechť $L_{abc} = \{a^n b^n c^n | n \geq 0 \}$. Platí, že $L_{abc} \notin \mathcal{L}_{CF}$.
\end{veta} 

\begin{proof} \label{dukaz} 
Důkaz se provede pomocí Pumping lemma pro bezkontextové jazyky, kdy ukážeme, že není možné, aby platilo, což bude implikovat pravdivost věty \ref{veta}.
\end{proof}

\section{Rovnice a~odkazy}

Složitější matematické formulace sázíme mimo plynulý text. Lze umístit několik výrazů na jeden řádek, ale pak je třeba tyto vhodně oddělit, například příkazem \verb|\quad|. 
 
$$\sqrt[x^2]{y^3_0} \quad \mathbb{N} = \{0, 1, 2,\ldots\} \quad x^{y^y} \neq x^{yy} \quad z_{i_j} \not\equiv z_{ij}$$

V~rovnici (\ref{eq:reference}) jsou využity tři typy závorek s~různou explicitně definovanou velikostí.

\begin{eqnarray}   \label{eq:reference}
 \bigg\{ \Big[ \big( a+b \big) *c \Big]^d +1 \bigg\} &= x\\
 \lim\limits_{x \rightarrow \infty} \displaystyle{\frac{\sin^2 x + \cos^2 x}{4}} &= y \nonumber
\end{eqnarray}

V~této větě vidíme, jak vypadá implicitní vysázení limity $\lim\nolimits_{a \to \infty}f(n)$ v~normálním odstavci textu. Podobně je to i~s~dalšími symboly jako $\sum^n_1$ či $\bigcup_{A \in \mathcal{B}}$ . V~případě vzorce $\lim\limits_{x \to \infty} \frac{\sin x}{x} = 1$ jsme si vynutili méně úspornou sazbu příkazem \verb|\limits|.

\begin{eqnarray}
 \int\limits_{a}^{b} f(x) dx & = & - \int_b^a f(x) dx \\ (\sqrt[5]{x^4})' = (x^{\frac{4}{5}})' & = & \displaystyle{\frac{4}{5}}x^{-\frac{1}{5}} = \displaystyle{\frac{4}{5\sqrt[5]{x}}} \\
 \overline{\overline{A \vee B}} & = & \overline{\overline{A} \wedge \overline{B}}
\end{eqnarray}

\section{Matice}

Pro sázení matic se velmi často používá prostředí \verb|\array| a~závorky (\verb|\left|, \verb|\right|). 

$$
\begin{pmatrix}
  \; {a+b} & {b-a} \; \\
  \; {\widehat{\xi+\omega}} & \hat{\pi} \; \\
  \; \vec{a} & \overleftrightarrow{AB} \; \\
  \; 0 & \beta \;
\end{pmatrix}
$$

$$A = 
\begin{vmatrix}\!
\begin{vmatrix}
  \; a_{11} & a_{12} & \cdots & a_{1n} \; \\
  \; a_{21} & a_{22} & \cdots & a_{2n} \; \\
  \; \vdots  & \vdots  & \ddots & \vdots \; \\
  \; a_{m1} & a_{m2} & \cdots & a_{mn} \;
\end{vmatrix}\!
\end{vmatrix}$$

$$
\begin{vmatrix}
\; t & u \; \\ \; v & w \; 
\end{vmatrix} = tw - uv
$$

Prostředí \verb|\array| lze úspěšně využít i~jinde.

$$\binom{n}{k} = 
\begin{cases}
     \; \frac{n!}{k!(n-k)!} & \quad \text{pro } 0 \leq k \leq n\\
     \; 0 & \quad \text{pro } k < 0 \text{ nebo } k > n\\
\end{cases}$$

\section{Závěrem}

V~případě, že budete potřebovat vyjádřit matematickou konstrukci nebo symbol a~nebude se Vám dařit jej nalézt v~samotném \LaTeX u, doporučuji prostudovat možnosti balíku maker \AmS-\LaTeX.
Analogická poučka platí obecně pro jakoukoli konstrukci v~\TeX u.

\end{document}
